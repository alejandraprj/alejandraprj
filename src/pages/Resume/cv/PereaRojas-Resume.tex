\documentclass[11pt]{article}
\usepackage[letterpaper, margin=0.5in, showframe=false]{geometry}
\usepackage[utf8]{inputenc}
\usepackage[T1]{fontenc}
\usepackage{lmodern}
\usepackage{mathptmx}
\usepackage{hyperref}
\usepackage{graphicx}
\usepackage{xcolor}
\usepackage[super]{nth}
\renewcommand{\labelitemi}{\textbullet}
\usepackage{enumitem}
\usepackage{fontawesome5}

\usepackage{aleCV}

\thispagestyle{empty}

\begin{document}

  \centerline{\Large{Alejandra J. Perea Rojas}}\vspace{1mm}

  \links

  % Education
  \title{Education}
    \edu
    {Harvard University}
    {Cambridge, MA}
    {Bachelor of Arts in Computer Science; Secondary in Physics.}
    {Anticipated: May 2024}
    { 
      Turing-complete \href{https://book.cs51.io/pdfs/abstraction-21-project.pdf}{OCaml interpreter},
      a C++ mini-OS \href{https://cs61.seas.harvard.edu/site/2021/WeensyOS/}{WeensyOS},
      a \href{https://cs61.seas.harvard.edu/site/2021/Shell/}{bash shell} in C++, 
      and a Python \href{http://procaccia.info/courses/CS182-F22/pset3.pdf}{Ghost AI}.
    }
    { 
      Systems Programming (CS61), Artificial Intelligence (CS182)
      Abstraction and Design in Computation (CS51), \newline
      Intro to Algorithms (CS120), Applied Linear Algebra (AM120), 
      Intro to Probability (STAT110).
    }

  \title{Experience}
    \job
    {Quantum Software Developer Intern}
    {May-Aug 2023}
    {IBM}
    {Yorktown Heights, NY}
    {
      \item Contributed to job compilation efficiency in Qiskit Runtime by implementing a Go endpoint for owner-verified deletion of compiled quantum payloads, complemented by a mock API for unit testing and to verify the functionality of the endpoint.
      \item Augmented Qiskit Runtime's reliability by designing over 20 error types and 30 unit tests. Implemented various forms of error propagation, logging, and handling across different Python codebases, including the Near-Time Compute services.
    }

    \job
    {Chief Technical Officer \& Co-Founder}
    {Apr-May 2023}
    {Entrevista}
    {New York, NY}
    {
      \item Created and launched the MVP of \href{https://www.entrevista.ai}{Entrevista} by developing a web-based JavaScript recorder and text-to-speech that simulates a real-time casing interview via a REST API framework and a Python video-processor, an open-source text-generator, and AWS S3 handlers. Deployed in the span of a month, leveraging Heroku, GitHub, and AWS.
      \item Earned a finalist position in the Y-Combinator application with two Co-Founders, initiating team and user acquisition.
    }

    \job
    {Software Engineer Intern}
    {Jun-Dec 2022}
    {Teamcore Lab at Harvard University}
    {Cambridge, MA}
    {
      \item Boosted efficiency and reliability of the PAWS SMART API by developing a Python HTTP API. This facilitated simultaneous multi-case testing, supplanted manual Postman HTTP requests, and yielded substantial time savings.
      \item Broadened the testing scope and enhanced error handling by leveraging Azure for data storage and creating 15 additional mock QGIS parks, enabling bulk request processing, thorough testing, and accelerating the integration of new features.
    }

    \job
    {Drones and Remote Sensors Intern}
    {May-Aug 2022}
    {Wildlife Conservation Society}
    {Cambridge, MA}
    {
      \item Developed an article processing interface with Python and SQL for Google News and Scholar. Contributed with over 1,500 parsed articles from the first use of the program, which helped identify quality sources to start the \href{https://library.wcs.org}{WCS Library}.
      \item With the parser algorithm, researched and drafted an 8-page \href{https://bit.ly/AI-Advancing-Video-Processing-and-CTDS}{white-paper} on state-of-the-art AI video processing and conservation tools that helped advance multiple projects at WCS throughout months after the internship.
    }
  
  % Activities
  \title{Activities}
    \job
    {Teaching Fellow}
    {Sep-Dec 2022}
    {Harvard University}
    {Cambridge, MA}
    {
      \item Facilitated a \href{https://cs61.seas.harvard.edu/site/2022}{Systems Programming} class 
      of about 200 students by section and office hours of 20+ students.
      \item Covered 6 topics using C++ and GNU: data memory and representation, assembly, kernel, caching, shell, and threading.
    }

    \job
    {DIB Advocacy Director}
    {Aug-Dec 2022}
    {Harvard Women in Computer Science}
    {Cambridge, MA}
    {
      \item Oversaw and collaborated with 8 members on initiatives to promote inclusion. With a budget of approximately \$300, helped organized events throughout the semester, providing resources and support to diverse groups of over 100 students.
    }

  % Skills
  \title{Skills}
  \skills
    {Proficient (3+ years) in Python and C++. Experienced (1+ years) with Go, Java, OCaml, and SQL.}
    {Proficient (3+ years) in HTML/CSS. Experienced (1+ years) with PHP, JavaScript, TypeScript, and React.}
    {Skilled with Azure, AWS, Docker, Qiskit, OpenAI, Linux, MATLAB, GNU, and Assembly.}
    {Fluent in Spanish (native), intermediate in Mandarin Chinese, and elementary in Japanese.}

\end{document} 

Other projects:
- Web-scraper for PCRI
- Finish Google Scholar API
- Web Lead for Harvard Pscyhedelics Club
- CS50, CS61, CS20, CS120, CS51, STAT110, MATH 21, AM120, PHYS 15, CS182, CS91R
- TODO: finish PHYS 15c, PHYS 143, CS124, CS161, ... 
